\documentclass[english, xcolor={table,usenames}]{beamer}
\usepackage{graphicx}
\usepackage[english]{babel}
\usepackage[utf8]{inputenx}
\usepackage[T1]{fontenc}      % Font encoding
\usepackage{palatino}          % lmodern font, correctly copyable characters in pdf
\usepackage{hyperref}         % Hyperlinks
\usepackage{amsmath}          % Math
\usepackage{amssymb}          % Math symbols
\usepackage{mathtools}        % Math tools
\usepackage{siunitx}          % SI units
\usepackage{microtype}        % Microtypography
\usepackage{bookmark}
\usepackage{subfigure}
\usepackage{tikz}
\usepackage{chngcntr}
\usepackage{ragged2e}
\counterwithin{subfigure}{figure}
\usepackage{caption}
\usepackage{listings}
\usepackage[binary-units]{siunitx}
\usepackage{tabularx}
\usepackage{booktabs}         % Tables
\usepackage{float}
\usepackage{listings}
\usepackage{bera}

%%%% COMPILE WITH LATEXMK %%%%

\usetheme[
  bullet=circle,                  % Use circles instead of squares for bullets
  titleline=false,                % Show a line below the frame
  alternativetitlepage=true,      % Use the fancy title
  titlepagelogo=logo-sapienza,    % Logo for the first slide
  watermark=watermark-diag,       % Watermark used in every slide
  watermarkheight=20px,           % Desired height of the watermark
  watermarkheightmult=6,          % Watermark image is actually x times bigger
  displayauthoronfooter=true,     % Display author name in the footer
]{Roma}
\watermarkoff%
\author{\textbf{Dario Loi} --- \textbf{1940849}}

\title{Multi-Lingual Natural Language Processing}
\subtitle{Homeworks Report}
\institute{M.Sc. in AI \& Robotics, \\ Sapienza, University of Rome.}
\date{A. Y. 2023--2024}

% Cover future items, do not cover past items

\setbeamercovered{transparent=20}

\colorlet{punct}{red!60!black}
\definecolor{background}{HTML}{EEEEEE}
\definecolor{delim}{RGB}{20,105,176}
\colorlet{numb}{magenta!60!black}
\lstdefinelanguage{json}{
    basicstyle=\tiny\normalfont\ttfamily,
    numbers=left,
    numberstyle=\tiny,
    stepnumber=1,
    numbersep=8pt,
    showstringspaces=true,
    breaklines=true,
    frame=lines,
    backgroundcolor=\color{background},
    literate=
     *{0}{{{\color{numb}0}}}{1}
      {1}{{{\color{numb}1}}}{1}
      {2}{{{\color{numb}2}}}{1}
      {3}{{{\color{numb}3}}}{1}
      {4}{{{\color{numb}4}}}{1}
      {5}{{{\color{numb}5}}}{1}
      {6}{{{\color{numb}6}}}{1}
      {7}{{{\color{numb}7}}}{1}
      {8}{{{\color{numb}8}}}{1}
      {9}{{{\color{numb}9}}}{1}
      {:}{{{\color{punct}{:}}}}{1}
      {,}{{{\color{punct}{,}}}}{1}
      {\{}{{{\color{delim}{\{}}}}{1}
      {\}}{{{\color{delim}{\}}}}}{1}
      {[}{{{\color{delim}{[}}}}{1}
      {]}{{{\color{delim}{]}}}}{1},
}


\begin{document}

\maketitle

\section{Homework 1-A}

\subsection{Introduction}
\begin{frame}{Tasks}
  For the first homework, I was assigned \alert{two} tasks:

  \begin{enumerate}
    \item WiC-ITA: Detect whether two italian words in two \alert{different} sentences are used with the \alert{same} meaning.
    \item ITAmoji: Predict which emoji was used in a given italian tweet.
  \end{enumerate}

  We will spend a few slides on each task.
\end{frame}


\begin{frame}[fragile]{WiC-ITA}
  The first task involved the parsing of JSONL files containing the Word-in-Context-ITA dataset, an example of which is shown below:


  \begin{figure}[H]
    \centering
    \begin{lstlisting}[language=json, escapeinside = {(*@}{@*)}, caption={Sample 23 from WiC-ITA \texttt{train.jsonl}.}]
  {
    "id": "lira.noun.15",
    "lemma": "lira",
    "sentence1": "In caso di inosservanza degli obblighi stabiliti dal comma 1 , si applica la sanzione amministrativa pecuniaria da lire dieci milioni a lire cento milioni .",
    "sentence2": "Per le finalit(*@\`{a}@*) di cui all' Art. 1 , comma 2 , della legge regionale 26 aprile 1995 , n. 31 recante \" Norme in materia di musei degli Enti locali e di interesse locale \" (*@\`{e}@*) autorizzata per l' esercizio finanziario 2000 la spesa di lire 200.000.000 .",
    "start1": 115,
    "end1": 119,
    "start2": 230,
    "end2": 234,
    "label": 1
  }
  \end{lstlisting}
  \end{figure}

\end{frame}

\begin{frame}[fragile]{Desired Format}

  The samples were rewritten for use by an LLM, as shown below:


  \begin{figure}[H]
    \centering
    \begin{lstlisting}[language=json, escapeinside = {(*@}{@*)}, caption={Same sample with required changes.}]
  {
    "id": "lira.noun.15",
    "lemma": "lira",
    "sentence1": "In caso di inosservanza degli obblighi stabiliti dal comma 1 , si applica la sanzione amministrativa pecuniaria da lire dieci milioni a lire cento milioni .",
      "sentence2": "Per le finalit(*@\`{a}@*) di cui all' Art. 1 , comma 2 , della legge regionale 26 aprile 1995 , n. 31 recante \" Norme in materia di musei degli Enti locali e di interesse locale \" (*@\`{e}@*) autorizzata per l' esercizio finanziario 2000 la spesa di lire 200.000.000 .",
    "start1": 115,
    "end1": 119,
    "start2": 230,
    "end2": 234,
    "choices": [
      "DIVERSO",
      "UGUALE"
    ],
    "label": 1
  }
\end{lstlisting}
  \end{figure}
\end{frame}

\begin{frame}[fragile]{ITAmoji}
  The second task also involved manipulation of JSONL files, a typical sample of ITAmoji is shown below:

  \begin{figure}[H]
    \centering
    \begin{lstlisting}[language=json, escapeinside = {(*@}{@*)}, caption={Sample 27 from \texttt{ITAmoji\_2018\_TRAINdataset\_v1.ANON.list}.}]
  {
    "uid": "447352763",
    "text_no_emoji": "... il rumore del mare \ufe0f #28Settembre <URL>",
    "created_at": "Thu Sep 28 15:32:06 +0000 2017",
    "label": "red_heart",
    "tid": "913426094002458626"
  }
  \end{lstlisting}
  \end{figure}

  For this task, we also have to add \alert{distractors}, that is, plausible alternatives to the correct label.

\end{frame}


\begin{frame}[fragile]{Desired Format... Again}
  The output from our distractor generation process on the previous sample is as follows:

  \begin{figure}[H]
    \centering
    \begin{lstlisting}[language=json, escapeinside = {(*@}{@*)}, caption={Sample 27 with generated distractors.}]
  {
    "id": "ITA-emoji-train-00000027",
    "sentence": "... il rumore del mare \ufe0f #28Settembre <URL>",
    "choices": [
        "red_heart",
        "two_hearts",
        "rose",
        "kiss_mark"
    ],
    "label": 0
  }
  \end{lstlisting}
  \end{figure}
\end{frame}

\begin{frame}{Distractor Clusters} % ["red_heart", "two_hearts", "blue_heart", "rose", "kiss_mark"],
  To generate distractors, we used a hand-crafted list of \alert{semantic clusters}, which we define as
  a set of emoji that are semantically related. For example, the cluster \texttt{love} contains the
  emojis \texttt{red\_heart}, \texttt{two\_hearts}, \texttt{blue\_heart}, \texttt{rose}, and \texttt{kiss\_mark}.

  When augmenting a sample, we select a cluster in which the correct label is present, and then randomly
  select three other labels from the same cluster. This way, we ensure that the distractors are
  capable to confuse the model.
\end{frame}

\begin{frame}{Distractor Sampling}
  To obtain a set of distractors from an ITAmoji sample, we follow this process:

  \begin{itemize}
    \item<1-> Randomly sample a cluster from the list of clusters that contain the correct emoji.
    \item<2-> Randomly sample three other emojis from the same cluster.
    \item<4-> Return the new sample with the correct label and the three distractors.
  \end{itemize}
\end{frame}

\section{Homework 1-B}

\end{document}
